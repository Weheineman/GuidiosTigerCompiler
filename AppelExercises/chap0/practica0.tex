\documentclass{article}
\usepackage[spanish]{babel}
\usepackage[latin1]{inputenc}
\usepackage{hyperref}
\usepackage{verbatim}
\usepackage{epsfig}
\usepackage{graphicx}

\usepackage{exercise}
\usepackage{listings}

\lstset{language=Ml}

\title{%
	Compiladores - LCC\\
	Pr�ctica 0. \\
	Introducci�n a Tiger
}
\date{}
\author{}
\begin{document}
\maketitle
%%%%%%%%%%%%%%%%%%%%%%%%%%%%%%%%%%%%%%%%%%%%%%%%%%%%%%%%%%%%%%
%\begin{Exercise}
%\end{Exercise}
%%%%%%%%%%%%%%%%%%%%%%%%%%%%%%%%%%%%%%%%%%%%%%%%%%%%%%%%%%%%%
\begin{Exercise}
    Escribir un ``Hola Mundo!'' en Tiger.
\end{Exercise}

\begin{Exercise}
    Las listas de enteros pueden representarse utilizando el siguiente record. 
    \begin{lstlisting}
    lista = {item: int, resto:lista}
    \end{lstlisting}
    Implemente las siguientes funciones:
\begin{itemize}
    \item \lstinline{cons} que recibe dos argumentos, un entero y una lista. Devuelve
        la lista formada por el entero seguido de la lista.
    \item \lstinline{length} que recibe una lista de enteros, y devuelve la longitud
        de la misma.
    \item \lstinline{concat} que recibe dos listas como argumentos y las concatena.
    \item \lstinline{snoc} que recibe dos argumentos: un entero $x$ y una lista $xs$.
        Devuelve la lista $xs$ con el elemento $x$ al final.
    \item \lstinline{filtra} que recibe un entero n y una lista l. Devuelve una lista formada por los elementos de l que no son iguales a n, manteniendo el orden de l.
    \item \lstinline{isin} que recibe dos argumentos: un entero $x$ y una lista $xs$.
        Devolviendo verdadero si $x$ aparece al menos una vez en $xs$.
    \item \lstinline{remove} que recibe dos argumentos: un entero $x$ y una lista
        $xs$. Devolviendo el resultado de borrar la primer aparici\'on de $x$ en
        $xs$ en el caso que $isin(x,xs)$ sea verdadero.
    \item \lstinline{removeall} que recibe dos argumentos similar a \verb1remove1
        pero que elimina todas las apariciones de $x$ en $xs$.
    \item \lstinline{reverse} que recibe una lista de enteros, y retorna la lista en
        orden inverso.
    \item \lstinline{printlist} que toma como argumento una lista de enteros y la
        muestra por pantalla.
\end{itemize}

\begin{lstlisting}
let
    type lista = {item:int, resto:lista}
    /* Definir cons         */
    /* Definir length       */
    /* Definir concat       */
    /* Definir snoc         */
    /* Definir filtra       */
    /* Definir isin         */
    /* Definir remove       */
    /* Definir removeall    */
    /* Definir reverse      */
    /* Definir printlist    */
in
    nil
end 
\end{lstlisting}
\end{Exercise}

\begin{Exercise}
    De forma similar a las listas de enteros del ejercicio anterior,
    podemos definir a los �rboles generales como:
\begin{lstlisting}
let
  /* define a tree */
  type tree = {key : int, children : treelist}
  type treelist = {hd : tree, tl : treelist}
in
   0
end
\end{lstlisting}

Escribir las siguientes funciones:
\begin{itemize}
    \item \lstinline{isBin} que dado un �rbol general responda con verdadero en el
        caso que sea un �rbol binario o falso si no lo es.
    \item \lstinline{isBComplete} que dado un �rbol general responda con verdadero en
        el caso que sea binario completo o falso en cualquier otro caso.
    \item \lstinline{printInOrder} que imprima todos los valores recorriendo el
        \'arbol en orden.
    \item \lstinline{printPosOrder} que imprima todos los valores recorriendo el
        \'arbol en orden.
\end{itemize}
\end{Exercise}

\begin{Exercise}
Definir las siguientes funciones:
\begin{itemize}
  \item \textbf{maxargs:} dado un \textit{tigerabs.exp} devuelve la m�xima cantidad de argumentos con
los cuales se ha llamado a la funci\'on \textit{print}.
  \item \textbf{cantplus:} dado un \textit{tigerabs.exp}, contar la cantidad de veces que se
utiliza la operaci\'on binaria de la suma.
\end{itemize}
\end{Exercise}

\begin{Exercise}
Encuentre el AST que corresponde a los siguientes fragmentos de c�digo. Puede usar la primera versi�n del compilador, comentando la l�nea

\verb1val _ = findEscape(expr)1
en \emph{tigermain.sml}.
\begin{itemize}
 \item [a)] \verb2a := 102
 \item [b)] \verb1for i := 0 to c do print (".")1
 \item [c)] \verb2f[a+1].data[0]2
 \item [d)]
 
 \begin{verbatim}
let
    var f := 10
in
    f(f, f); f
end
 \end{verbatim}
 
 \item [e)] \verb1type lista = {item:int, resto:lista}1
 \item [f)] \verb1if row[r]=0 & a<b then g(r)1
\end{itemize}

�Todos los fragmentos pueden ser parte de un programa v�lido? �Por qu� hay un problema al copiar directamente el fragmento \emph{e}? Descomente la l�nea comentada anteriormente. �Qu� error detecta ahora el compilador?
\end{Exercise}

\begin{Exercise}
% \section{}
Encuentre el c�digo que genera los siguientes ASTs. No tome en cuenta el valor del campo \verb1pos1.

\begin{itemize}
 \item [a)] \verb2ArrayExp({init = IntExp(5, 0), size = IntExp(10, 0), typ = "a"}, 0)2
 \item [b)] \verb1VarExp(SubscriptVar(SimpleVar "a", IntExp(7, 0)), 0)1
 \item [c)] \verb1AssignExp({exp = NilExp 0, var = SimpleVar "a"}, 0)1
\end{itemize}
\end{Exercise}

\begin{Exercise}
    Implementar \textit{pretty printers} para los siguientes \textit{datatypes}:
    \begin{itemize}
        \item \verb1EnvEntry1
        \item \verb1Tipo1
    \end{itemize}
    Implementar la funci�n 
    \begin{lstlisting}
val tabPrint : ('a -> string) * ('b -> string) * ('a, 'b) Tabla -> string
    \end{lstlisting}
    % \lstinline{val tabPrint : ('a -> string) * ('b -> string) * ('a, 'b) Tabla    -> string}.
\end{Exercise}

\end{document}
